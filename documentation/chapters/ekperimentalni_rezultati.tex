\chapter{Eksperimentalni rezultati}
Uspoređivanjem tablica \ref{res:ref_est_table} i \ref{res:ref_est_table_vox} može se doći do nekoliko zaključaka. Drugi algoritam je bolje estimirao točke i s time bolje estimirao ukupan prijeđeni put naspram prvog algoritma. S obzirom na estimacije x i y koordinata prvi algoritam ima nešto veću srednju pogrešku naspram drugog algoritma. Iako se može zanimariti vrijednost z koordinate, može se vidjeti da drugi algoritam i dalje ima manju srednju pogrešku od prvog algoritma. Oba algoritam imaju slične srednje pogrške za valjanje i poniranje, dok drugi algoritam ima nešto manju grešku prilikom računanja skretanja.

Može se zaključiti na temelju ovih ekspreimantalnih rezultata da bez obzira što smo u drugome algoritmu smanjili broj točaka u oblaku on je u večini slučaja estimirao bolje podatke. Također je vrijeme trajanja drugog algoritma bilo brže. To omogućuje stvaranje algoritama koji imaju bolju ili jednaku točnost od trenutnih, a radi mnogo brže i na manje podataka.
