\section{Lokalizacija}

Roboti i vozila u većini slučajeva se primjenjuju za izvođenje repetitivnih ili opasnih po život poslova. Čovjeka zamjenjuje robot ali to znači da je za upravljanje robota zadužen taj isti robot ili neki udaljeni sustav, čovjek više nema tu ulogu. U tu svrhu su roboti i vozila opremljeni raznim senzorima da bi se to omogućilo. Svi podaci prikupljeni iz tih senzora se koriste prilikom lokalizacije robota ili vozila.

Lokalizacija je postupak određivanja lokacije objekta u prostoru iz senzorskih podataka. Lokalizacija može biti vrlo zahtjevan zadatak te se u tu svrhu mogu koristiti algoritmi različitih složenosti. Što je algoritam složeniji, to se sporije izvodi ali je točniji dok se neki više optimizirani tj. brži algoritmi brže izvode ali se događaju veće greške te se one akumuliraju prilikom izvođenja algoritma.

Koriste se algoritmi za istovremenu lokalizaciju i mapiranje \cite{Simultaneous_localization_and_mapping} tj. za stvaranje karte nepoznatog prostore kojime se robot kreće te koordinate u tome prostoreu. Lokalizacija odgovara `Gdje je robot sada?' tj. gdje je sada naspram prethodne lokacije. Na to pitanje se može odgovoriti ovisno o tome radi li se o lokalizaciji u otvorenome ili zatvorenome prostoru. Lokacija robota je oglavnom prikazana u kartezijskom koordinatnom sustavu, bilo to u 2d ili 3d prostoru.

Postoje dvije vrste lokalizacije:

\begin{itemize}
  \item Lokalna - informacije se prikupljaju pomoću senzora robota iz njegove okoline
  \item Globalna - informacije se dobiju iz GPS-a ili slično
\end{itemize}

\newpage
\subsubsection{Pregled pristupa problemu lokalizacije}

Jedna od najjednostavnijih metoda je "Metoda najmanjih kvadrata" (eng. Least Squares Error) gdje se koristi metoda najmanjih kvadrata za regresijsku analizu podataka. Cilj te metode jest minimizacija pogreške gdje robot jest i gdje bi robot trebao biti tj. ona okvirno procjenjuje gradijent funkcije pomaka robota.

Praćenje pozicije (eng. Pose Tracking) metoda se koristi kada je poznata početna pozicija robota pa je potrebno samo pratiti njegovu poziciju kroz vrijeme. Metoda koristi ekstrakciju tj. izdvajanje značajki okoline koje se mogu uspoređivati te se tako kroz vrijeme može pratiti promjena položaja nekih uočljivih objekata.

Metoda višestrukih hipoteza (eng. Multiple Hypotesis Localization) pretpostavlja da početna pozicija nije poznata ali je poznata topografija mape. U ovome slučaju početnu poziciju može robotu pridodati korisnik ili robot uvijek može započeti iz iste pozicije. Ideja iza ove metode je da se detektira svojstvo te se preko njega stvaraju hipoteze o položaju robota naspram toga objekta kojemu pripada to svojstvo. Može se stvoriti nova hipoteza ili se može poboljšati neka od prethodnih hipoteza ili ipak eliminirati.

Metoda iteracije najbližih točaka (eng. Iterative Closest Point) minimizira razliku između dvije skupine točaka tako da iterira između svake dvije točke te pronalazi onu kombinaciju koja daje najmanju grešku. Često se koristi pri rekonstrukciji 2D ili 3D površina nakon skeniranja. Tijekom izvođenja te metode jedna skupina točaka je fiksna tj. referentna dok se druga transformira tako da se najbolje slažu koordinatama u referentnom skupu. Postoje mnoge varijante ICP-a od kojih su point-to-point (usporedba točka-točka) , point-to-plane (usporedba točke-površina) i point-to-line (usporedba točka-linija) najpopularnije.

Metoda usporedbe očitanja (eng. scan matching) koristi dva uzastopna očitanja senzora robota poput lasera, sonara, … da se pronađe relativan pomak robota u prostoru. Razlike između dva očitanja senzora se mogu uočiti vrlo lako zbog učestalosti skeniranja tj. frekvencije dohvaćanja senzorskih podataka te o gustoći lasera kojih uglavnom ima od nekoliko stotina do nekoliko tisuća. Načina na koji se zapravo traže razlike između dva očitanja ima mnogo. Koriste se laseri (eng. laser range finders) da bi vidio prepreke i odometrija kotača (eng. wheel odometry) da dobije okvirno stanje robota. Odometrija iz kotača ima određenu grešku zbog proklizavanja kotača ili nekog drugog razloga te se ona tada ispravlja pomoću izračunatih vrijednosti odometrije iz lasera. Ova metoda je zapravo metoda iteracije najbližih točaka ali ograničena u dvije dimenzije.

