\chapter{Uvod}

U zadnjih 30ak godina se može vidjeti ubrzan napredak u automatici i računarstvu. Tome su prethodile godine teorizacija iz kojih su kasnije nastali razni algoritmi za optimizacije i obradu podataka. Sada s razvojem računalnih aspekata poput memorija te procesorske snage možemo obrađivati sve više podataka u što manje vremena. To omogućava današnjim vozilima da u potpunosti budu električna što znači da su stalno u kontaktu s okolinom te mogu bez prestanka skupljati podatke iz okoline. Ta vozila su zapravo skupine senzora i urađaja konstantno povezanih na internet.

Skoro sva takva vozila danas imaju mogućnost autonomne vožnje. To omogućuju ljudima ugodnije iskustvo te veliko smanjenje broja prometnih nesreća koje uzrokuju vozači u stanjima koja nisu prigodna za vožnju ili čistome nemaru. S obzirom da je sigurnost u prometu jedna od najvažnijih stvari po pitanju svih sudionika ,ona mora biti prioritet. To znači da svi algoritmi za bilo kavo upravljanje vozilom moraju biti u potpunosti testirani te bez ikakvih pograšaka. Naravno to je nemoguć zahtjev zato što u takvoj domeni ti algoritmi za rad imaju previše varijabli koje se ne mogu uzeti u obzir. 

Cilj ovoga rada je ukratko objesniti rad i prkazati rezultate nekoliko algoritama te njihove točnosti. Ulaz u algoritam su senzorska očitanja dok su izlazi lokacija vozila tj. relativna promjena lokacije između dva očitanja. Za usporedbu rezultata koristimo referentne podatke koji su prikupljni iz simulatora te je tako garantirana njihova točnost. Nekoliko primjera podataka je provedeno kroz algoritme te uspoređeno s referentnim podacima. Rezultati tih evaluacija su ilustrirani pomoću grafova. 
