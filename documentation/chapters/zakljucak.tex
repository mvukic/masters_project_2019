\chapter{Zaključak}
Prilikom lokalizacije vozila ili robota točnost podataka je vrlo važna. Koriste se podaci iz senzora, najčešće LIDAR-a da bi se odredio relativan položaj vozila. Ti podaci se prikazuju kao skupovi točaka. Ti podaci su vrlo jednostavni te ih je zato vrlo lako obrađivati u računalu. Oni se obrađuju raznim algoritmima, ali ti algoritmi ne mogu biti u potpunosti točni. Točnost algoritama može biti promjenjena zbog okoline u kojoj se vozilo nalazi, šuma u očitanjima, broju reprezentativnih podataka u očitanju, samih grešaka u fizičkoj izvedbi snezora ili zbog načina na koji algoritam radi. Za evaluaciju točnosti algoritama za različite tipove skupova točaka se koristi simulator kao izvor referentnih podataka. U ovome radu su evaluiran je generalizirajući ICP algoritam i ICP s grupiranjem točaka na podskupovima estimiranih lokacija. Rezultati su prikazani s grafovima te podacima o relativnoj i srednjoj pogrešci. Zbog same prirode simulatora, referentni podaci su garantirano točni te to omogućuje testiranje i drugih algoritama koji koriste senzore osim lasera.