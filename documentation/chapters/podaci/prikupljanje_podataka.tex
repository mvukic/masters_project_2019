\section{Prikupljanje podataka}

Podaci su prikupljeni tako da se simulator pokrene u poslužiteljkom načinu rada te se tada pokreće klijentska skripta napisana u python jeziku. Ta skripta uspostavi kontakt s poslušiteljem te se tako šalju naredbe. Te naredbe će zapravo stvoriti naše vozilo, ostale sudionike i senzor. Nakon što smo prikupili dovoljno podataka skripta će obrisati stvorene objekte i spremiti podatke u datoteke. Tada simulator može prekinuti s radom te se ti podaci mogu obrađivati na bilo koji način.

\subsubsection{Pokretanje simulatora}

\begin{listing}[!ht]
  \begin{minted}{bash}
    CarlaUE4.exe \ 
      /Game/Carla/Maps/Town01 \
      -quality-level=Low \
      -benchmark -fps=15 \
      -windowed -ResX=800 -ResY=600 \
      -carla-port=2000 \
  \end{minted}
  \caption{Carla naredba}
  \label{coderef:carla_start}
\end{listing}

Simulator Carla se pokreće pomoću python skripte zbog boljeg upravljanja parametrima ali  se zapravo sastoji od naredbe pokazane u primjeru \ref{coderef:carla_start}. Simulacija se pokreće u mapi pod nazivom Town01. Kvaliteta je postavljena na najnižu vrijednost kao i broj slika u sekundi zbog boljih performansi izvođenja. Prozor smo postavili na vrlo malu rezoluciju od 800 pixela širine i 600 pixela visine također zbog boljih performansi. Vrlo važan parametar je sučelje preko kojega klijentski program komunicira s poslužiteljem. Ovdje je definiran kao 2000.


\subsubsection{Klijentska kripta}
Referentni i testni podaci su prikupljeni iz simualtora ali iz različitih izvora. Referentni podaci su prikupljeni iz samoga simulatora dok su testni podaci prikupljeni pomoću senzora.

\begin{listing}[!ht]
  \begin{minted}[frame=lines, linenos]{python}
  class CarlaProp:
    spawn_delay = 1.0
    host = "localhost"
    port = 2000
  \end{minted}
  \caption{Carla postavke}
  \label{coderef:carla_properties}
\end{listing}

\begin{listing}[!ht]
  \begin{minted}[frame=lines, linenos]{python}
  def connect_to_carla(self):
    self.client = carla.Client(CarlaProp.host, CarlaProp.port)
    self.client.set_timeout(2.0)
    self.world = self.client.get_world()
    settings = self.world.get_settings()
    settings.synchronous_mode = True
    self.world.apply_settings(settings)
  \end{minted}
  \caption{Uspostava konekcije s poslužiteljem}
  \label{coderef:carla_connect}
\end{listing}

U primjeru izvornoga koda \ref{coderef:carla_connect} klijent se spaja na Carla poslužitelj čiju smo lokaciju (IP adresu i sučelje) definirali u klasi \mintinline{python}{CarlaProp}. Također postavljamo sinkroni način rada simulatora, a razlog je taj što želimo upravljati frekvencijom slanja podataka iz poslužitelja prema klijentima. Varijabla \mintinline{python}{self.world} služi za izvođenje svih operacija koje su vezane uz svijet.

Svaka mapa ima već unaprijed definirane točke stvaranja tj. koordinate u svijetu na kojima možemo stvoriti objekte. Te koordinate se nalaze na cestama. Njih možemo dobiti naredbom prikazanom na primjeru izvornoga koda \ref{coderef:spawn_points}. 

\begin{listing}[!ht]
  \begin{minted}[frame=lines, linenos]{python}
    def get_spawn_points(world):
      return list(world.get_map().get_spawn_points())
  \end{minted}
  \caption{Dohvaćanje liste koordinata stvaranja}
  \label{coderef:spawn_points}
\end{listing}

Sljedeće što slijedi je stvaranje ostalih sudionika prometa tj. ostalih vozila. Carla ima već unaprijed definirane nacrte raznih objekata.

\begin{listing}[!ht]
  \begin{minted}[frame=lines, linenos]{python}
def get_vehicle_blueprints(world):
  blueprints = world.get_blueprint_library().filter('vehicle.*')
  blueprints = [x for x in blueprints if int(x.get_attribute('number_of_wheels')) == 4]
  return [x for x in blueprints if not x.id.endswith('isetta')]
  \end{minted}
  \caption{Dohvaćanje nacrta vozila}
  \label{coderef:vehicle_blueprints}
\end{listing}

Na \ref{coderef:vehicle_blueprints} se vidi kako koristimo knjižnicu nacrta da bi filtrirali nama potrebne nacrte. Koristiti će se samo vozila koja imaju 4 kotača.

\begin{listing}[!ht]
  \begin{minted}[frame=lines, linenos, escapeinside=!!]{python}
  def spawn_npcs(self):
    blueprints = utils.get_vehicle_blueprints(self.world)
    points = self.spawn_points[1:self.npc_number+1]
    for i in range(self.npc_number):
      actor_blueprint = random.choice(blueprints)
      print(f"\t{actor_blueprint}")
      actor_spwn_point = points[i]
      spawned_actor = self.world.try_spawn_actor(actor_blueprint, actor_spwn_point) !\label{lineref:spawn_instance}!
      spawned_actor.set_autopilot() !\label{lineref:set_autopilot}!
      self.npcs.append(spawned_actor)
    self.tick()
  \end{minted}
  \caption{Stvaranje ostalih vozila}
  \label{coderef:vehicle_spawning}
\end{listing}

Koristeću točke stvaranja i nacrte vozila sada se mogu ta vozila stvoriti u svijetu. Na \ref{coderef:vehicle_spawning} se koristeći petljom stvara unaprijed zadan broj ostalih sudionika definiranih u varijabli klase \mintinline{python}{self.npc_number}. Njihove reference se tada spremaju u listu zato što se na kraju izvođenja moraju uništiti. Stvaranje instance nacrta se izvodi naredbom na liniji \ref{lineref:spawn_instance}. Također smo svakoj instanci definirali autonomni način rada na liniji \ref{lineref:set_autopilot}.

Sada se definira vozilo koje zapravo promatramo tj. koje ima na sebi lidar senzor. To se radi na približno jednak način kao u primjeru \ref{coderef:vehicle_spawning}. Izvorni kod je prikazan na primjeru \ref{coderef:actor_spawning}.


\begin{listing}[!ht]
  \begin{minted}[frame=lines, linenos]{python}
  def spawn_actor(self):
    spawn_point = self.spawn_points[0]
    actor_blueprint = utils.get_vehicle_blueprint(self.world.get_blueprint_library())
    print(f"\t{actor_blueprint}")
    self.actor = self.world.spawn_actor(actor_blueprint, spawn_point)
    self.actor.set_autopilot()
    self.tick()
  \end{minted}
  \caption{Stvaranje promatranoga vozila}
  \label{coderef:actor_spawning}
\end{listing}

Sada slijedi pronalazak nacrta za lidar senzor, postavljanje njegovih atributa, njegovo instanciranje i postavljanje na promatrano vozilo. Izvorni kod je prikazan na \ref{coderef:lidar_find_spawn}. Postavke LIDAR senzora se nalaze u klasi \mintinline{python}{LIDARProp}.


\begin{listing}[!ht]
  \begin{minted}[frame=lines, linenos]{python}
  class LIDARProp:
    sensor_tick = str(0.0)
    channels = str(360)
    laser_range = str(1500.0)
    rotation_frequency = str(20.0)
    points_per_second = str(600_000)
    upper_fov = str(45.0)
    lower_fov = str(-80.0)
    location = carla.Transform(carla.Location(x=0, y=0, z=4))
  \end{minted}
  \caption{LIDAR atributi}
  \label{coderef:lidar_props}
\end{listing}

\begin{listing}[!ht]
  \begin{minted}[frame=lines, linenos, escapeinside=!!]{python}
    def connect_LIDAR(self):
      blueprint = utils.get_lidar_sensor_blueprint(self.world.get_blueprint_library()) !\label{lineref:get_lidar_blueprint}!
      blueprint.set_attribute('sensor_tick', LIDARProp.sensor_tick) !\label{lineref:lidar_prop_start}!
      blueprint.set_attribute('channels', LIDARProp.channels)
      blueprint.set_attribute('range', LIDARProp.laser_range)
      blueprint.set_attribute('rotation_frequency', LIDARProp.rotation_frequency)
      blueprint.set_attribute('points_per_second', LIDARProp.points_per_second)
      blueprint.set_attribute('upper_fov', LIDARProp.upper_fov)
      blueprint.set_attribute('lower_fov', LIDARProp.lower_fov) !\label{lineref:lidar_prop_end}!
      utils.print_sensor_blueprint_data(blueprint)
      self.lidar = self.world.try_spawn_actor(blueprint, LIDARProp.lidar_relative_postion, attach_to=self.actor)  !\label{lineref:connect_lidar}!
      self.lidar.listen(lambda data: self.lidar_callback(data))  !\label{lineref:lidar_callback_set}!
      self.tick()
  \end{minted}
  \caption{Stvaranje LIDAR senzora}
  \label{coderef:lidar_find_spawn}
\end{listing}

Na liniji \ref{lineref:get_lidar_blueprint} dohvaćamo nacrt lidar senzora. Tada od linije \ref{lineref:lidar_prop_start} do \ref{lineref:lidar_prop_end} postavljamo zadane postavke nad nacrtom. Konačno na liniji \ref{lineref:connect_lidar} stvaramo instancu senzora ali metodi predajemo dodatan parametar \mintinline{python}{attach_to} koji je je jednak referenci na naše vozilo. Također umjesto stvarnih koordinata, za lokaciju senzora postavljamo lokaciju relativnu naspram lokacije vozila. NA liniji \ref{lineref:lidar_callback_set} postaljamo metodu \mintinline{python}{lidar_callback()} kao metodu koju će simulator pozvati svaki puta kada senzor očita okolinu i pošalje podatke.
Spremanje podataka se izvršava tek nakon što smo sakupili konačan broj očitanja. Za spremanje podataka u datoteke se koristi posebna klasa \mintinline{python}{DataSaver} pokazana na primjeru \ref{coderef:data_saver}.

\begin{listing}[!ht]
  \begin{minted}[frame=lines, linenos, escapeinside=!!]{python}
class DataSaver:

  def __init__(self):
    self.pc_path = 'output/point_clouds'
    self.trans_path = 'output/actor_transforms'
    self.rel_path = 'output/relative_transform.txt'

  def initialize_folders(self):
    utils.create_directory(self.trans_path)
    utils.create_directory(self.pc_path)

  def save(self, scans):
    self.initialize_folders()
    self.save_rel_trans()
    start_time = time.time()

    with ThreadPoolExecutor(max_workers=10) as executor:
      jobs = list()
      for i, (scan, transform) in enumerate(scans[1:]):
        job = executor.submit(self.process_pair, scan, transform, i)
        jobs.append(job)
      for future in as_completed(jobs):
        future.result()

  \end{minted}
  \caption{Klasa za spremanje podataka}
  \label{coderef:data_saver}
\end{listing}

\begin{listing}[!ht]
  \begin{minted}[frame=lines, linenos, escapeinside=!!]{python}
  def process_pair(self, scan, trans, i):
    scan_path = f'{self.pc_path}/{scan.frame_number:06d}.ply'
    self.save_scan(scan, scan_path, i + 1)
    trans_path =  f'{self.trans_path}/{scan.frame_number:06d}.txt'
    self.save_transform(trans, scan.timestamp, trans_path, i + 1)

  def save_rel_trans(self):
    with open(self.rel_path, 'w+') as file:
      file.write(utils.transform_to_string(LIDARProp.location))

  def save_scan(self, scan, path, i = 0):
      scan.save_to_disk(path)

  def save_transform(self, trans, timestamp, path, i = 0):
    with open(path, 'w+') as file:
      content = f"{timestamp}\n{utils.transform_to_string(trans)}"
      file.write(content)
  \end{minted}
  \caption{Klasa za spremanje podataka - nastavak}
  \label{coderef:data_saver_cont}
\end{listing}

Datoteke s informacijama o lokaciji vozila se sastoje od 2 reda. Prvi red sadrži vremensku oznaku a drugi sadrži lokaciju i transformaciju koji su opisani u prijašnjem poglavlju. Datoteke koje sadrže podatke o jednome očitanju se nalaze u tekstualnim datotekama s ekstenzijom .ply te se sastoji od zaglavlja i po jedan redak za svaku točku u očitanju. Također postoji još jedna datoteka koja samo sadrži relativnu transformaciju između senzora i vozila. Ona se nalazi u direktoriju output.