\section{Simulator}

Za točne referentne podatke potrebno je imati simulirano okruženje. Takvo simulirano okruženje se zove simulator. Potreban je simulator koji već ima integrirane razne mape, razne senzore, vozila te način komunikacije s tim vozilima iz vanjskih skripti. Neki od simulatora su opisani u sljedećem tekstu.

\subsubsection{Carla}
\begin{figure}[ht!]
  \centering
  \includegraphics[scale=0.5]{images/carla_logo.png}
  \caption{Carla logo\cite{logo:carla}}
\end{figure}

Carla\cite{dosovitskiy17} je simulacijsko okruženje koje služi za testiranje metoda i algoritama prilikom razvoja autonomnih vozila. U pozadini koristi Unreal Engine za izvršavanje simulacije. Simulator se ponaša kao poslužitelj koji prima naredbe iz vanjskih klijentskih programa. Ti klijentski programi su pisani u programskom jeziku pytohn.
Carla ima integirane razne senzore te su neki od njih:
\begin{itemize}
  \item RGB kamera
  \item LIDAR senzor
  \item Senzor dubine
  \item GNSS
\end{itemize}

Sponzori projekta su Intel, Toyota, GM i Computer vision Center. Više o ovome simulatoru će biti u sljedećem poglavlju.

\subsubsection{Apollo}
\begin{figure}[ht!]
  \centering
  \includegraphics[scale=0.2]{images/apollo_logo.png}
  \caption{Apollo logo\cite{logo:apollo}}
\end{figure}

Apollo je također rješenje za testiranje autonomnoh vozila. Sadrži simulator ali također je i potpuno komercijalno rješenje. Podržava razne scenarije, ima sustav ocjenjivanja koji daje ocjenu na temelju desetak metrika. Simulacije zapravo provodi u oblaku tj. koristi Microsoft Azure. Sponzori projekta su mnoge azijske tvrtke kao i Ford, Microsoft, Daimler, Honda, Intel i ostali.

\subsubsection{rFpro}
\begin{figure}[ht!]
  \centering
  \includegraphics[scale=0.2]{images/rfpro_logo.jpg}
  \caption{rFpro logo\cite{logo:rfpro}}
\end{figure}

rFpro je kompletno rješenje za testiranje autonomnih vozila. U potpunosti je kommercijalno rješenje ali je zato jedno od najboljih u svijetu. Uglavnom je usredotočeno na primjenu strojnog učenja u autonomnim vozilima. Ima jednu od najvećih baza digitaliziranih stvarnih likacija diljem svijeta. Dinamički sustav vremena omogućuje testiranje ponašanja vozila u raznim vremenskim uvjetima. Sponzori projekta su BMW, Shell, GM, Renault i ostali.

\subsubsection{AVSimulation}
\begin{figure}[ht!]
  \centering
  \includegraphics{images/avsimulation_logo.png}
  \caption{AVSimulation logo\cite{logo:avsimulation}}
\end{figure}

AVSimulation se zapravo sastoji od simulatora vožnje i samog simulatora SCANeR. SCANeR je skup aplikacija koji pružaju rute, senzore, vozila, dinamičko vrijeme, pisanje skripti. Vrlo je modularan. Smulator vožnje je zapravo kupola koja se sastoji od cijelog vozila te se zapravo kretanje tog vozila simulira unutar te kupole. Sponzori su Renault, PSA, Volvo, Microsoft, Mazda i ostali.
\newpage