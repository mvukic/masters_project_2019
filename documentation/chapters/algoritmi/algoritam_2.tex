\section{ICP algoritam s grupacijom točaka}

\subsubsection{Opis algoritma}
Koristi se metoda iteracije najbližih točaka s grupiranjem točaka\cite{pcl:voxelgrid}. Radi tako da se cijeli oblak točaka podijeli na kocke zadane veličine te se tada unutar te kocke filtriraju točke na temelju njihove centroide\cite{wiki:Centroid}. U jednome od očitanja iz primjera A se broj točaka s 35222 smanjio na 1548 što značajno ubrzava ICP algoritam. Ovaj algoritam zapravo aproksimira skup točaka samo jednom točkom. U primjeru \ref{coderef:voxel_grouping} se vidi da je parametar visine, širine i duljine kocke uvijek jednak i iznosi 1 centimetar što znači da će se oblak točaka podijeliti na kocke veličine 1 centimetar. Performanse toga algoritma kao i točnost reprezentacije originalnog oblaka ovise o tom parametru kao i gustoći točaka.


\subsubsection{Izvorni kod algoritma}
Kod je vrlo sličan kao i u prethodnome algoritmu samo što sada imamo jedan korak više prije obrade oblaka ICP algoritmom. Doajemo linije \ref{lineref:voxel1} i \ref{lineref:voxel2} s pozivima metode \ref{coderef:voxel_method}. Unutar nje se inicijalizira objekt tipa \mintinline{text}{VoxelGrid<PT>}. Kao parametre prima oblak točaka i veličinu područja na koje će podijeliti oblak točaka.

\begin{listing}[H]
  \begin{minted}[frame=lines, linenos]{text}
void downsample_using_voxel_grid(
  PointCloudType::Ptr& cloud,
  float width, float height, float length,
  PointCloudType::Ptr& downsampled
 ) {
  VoxelGrid<PT> vg;
  vg.setInputCloud(cloud);
  vg.setLeafSize(width, height, length);
  vg.filter(*downsampled);
}
  \end{minted}
  \caption{Metoda za grupaciju točaka}
  \label{coderef:voxel_method}
\end{listing}

\begin{listing}[H]
  \begin{minted}[frame=lines, linenos, escapeinside=!!]{text}
void process_files(vector<path> paths, ICP icp) {
 for (int i = 0; i < paths.size() - 1; i++) {
  if (i == 0) {
   load_point_cloud(paths.at(i).string(), *cloud_ref);
  }
  load_point_cloud(paths.at(i + 1).string(), *cloud_target);

  string first = paths.at(i).stem().string(); 
  string second = paths.at(i + 1).stem().string();

  if (i == 0) {
   downsample_using_voxel_grid(
     cloud_ref, 1.0f, cloud_ref_filtered !\label{lineref:voxel1}!
   );
  }
  downsample_using_voxel_grid(
    cloud_target, 1.0f, cloud_target_filtered !\label{lineref:voxel2}!
  );

  icp.setInputCloud(cloud_ref_filtered);
  icp.setInputTarget(cloud_target_filtered);
  icp.align(*cloud_reg);

  if (icp.hasConverged()) {
   save_matrix(icp, first, second);
  }
  *cloud_ref = *cloud_target;
 }
}
  \end{minted}
  \caption{ICP grupacija točaka - obrada oblaka}
  \label{coderef:voxel_grouping}
\end{listing}
