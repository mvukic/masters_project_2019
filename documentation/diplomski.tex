\documentclass[times, utf8, diplomski]{fer}
\usepackage{booktabs}
\usepackage{subcaption}
\usepackage{amsmath}
\usepackage{pythonhighlight}
\usepackage{xcolor}
\usepackage{listings}
\usepackage{minted}
\usepackage{algorithm2e}

% Promjeni listing caption
\renewcommand{\listingscaption}{Izvorni kod}

\begin{document}

% TODO: Navedite broj rada.
\thesisnumber{1935}

% TODO: Navedite naslov rada.
\title{Lokalizacija autonomnog vozila u simuliranom urbanom okruženju}

% TODO: Navedite vaše ime i prezime.
\author{Matija Vukić}

\maketitle

% Ispis stranice s napomenom o umetanju izvornika rada. Uklonite naredbu \izvornik ako želite izbaciti tu stranicu.
%\izvornik
\begin{center}
  \makebox[\textwidth]{\includegraphics[width=\paperwidth]{images/izvornik.png}}
\end{center}

% Dodavanje zahvale ili prazne stranice. Ako ne želite dodati zahvalu, naredbu ostavite radi prazne stranice.
\zahvala{Zahvala mentoru Doc. Dr. Sc. Ivanu Markoviću za strpljenje i vodstvu}

\tableofcontents
\chapter{Uvod}
Uvod rada. Nakon uvoda dolaze poglavlja u kojima se obrađuje tema. Cite one: \textit{The \LaTeX\ companion} \cite{oetiket2007lshort}. Nakon uvoda dolaze poglavlja u kojima se obrađuje tema. Nakon uvoda dolaze poglavlja u kojima se obrađuje tema. Nakon uvoda dolaze poglavlja u kojima se obrađuje tema. Nakon uvoda dolaze poglavlja u kojima se obrađuje tema. Nakon uvoda dolaze poglavlja u kojima se obrađuje tema. Nakon uvoda dolaze poglavlja u kojima se obrađuje tema.
Nakon uvoda dolaze poglavlja u kojima se obrađuje tema. Hehe.
Cite two \cite{ungar2002uvod} and another book \cite{downes2002shortams}. %ok

\chapter{Problem lokalizacije vozila}
\section{Lokalizacija}

Roboti i vozila u većini slučajeva se primjenjuju za izvođenje repetitivnih ili opasnih po život poslova. Čovjeka zamjenjuje robot ali to znači da je za upravljanje robota zadužen taj isti robot ili neki udaljeni sustav tj. čovjek više nema tu ulogu. U tu svrhu su roboti i vozila opremljeni raznim senzorima da bi se to omogučilo. Svi podaci prikupljeni iz tih senzora se koriste prilikom lokalizacije robota ili vozila.

Lokalizacija je postupak određivanja lokacije objekta u prostoru iz ulaznih podataka. Lokalizacija može biti vrlo zahtjevan zadatak te se u tu svrhu mogu koristiti algoritmi različitih složenosti. Što je algoritam složeniji to se sporije izvodi ali je točiji dok se neki više optimizirani tj. brži algoritmi brže izvode ali postoji veća vjerojatnost da je došlo do pogreške prilikom izvođenja.

Koriste se algoritmi za istovremenu lokalizaciju i mapiranje [dodaj SLAM link] tj. za stvaranje karte nepoznatog prostore kojime se robot kreće te koordinate u tome prostoreu. Lokalizacija odgovara `Gdje je robot sada?' tj. gdje je sada naspram prethodne lokacije. Na to pitanje se može odgovoriti ovisno o tome radi li se o lokalizaciji u otvorenome ili zatvorenome prostoru. Lokacija robota je oglavnom prikazana u kartezijskom koordinatnom sustavu, bilo to u 2d ili 3d prostoru.

Postoje dvije vrste lokalizacije:

\begin{itemize}
  \item Lokalna - informacije se prikupljaju pomoću senzora robota iz njegove okoline
  \item Globalna - informacije se dobiju iz GPS-a ili slično
\end{itemize}

\newpage
\subsubsection{Neke metode lokalizacije}

Jedna od najjednostavnijih metoda je "Metoda najmanjih kvadrata" (eng. Least Squares Error) gdje se koristi metoda najmanjih kvadrata za regresijsku analizu podataka. Cilj te metode jest minimizacija pogreške gdje robot jest i gdje bi robot trebao biti tj. ona okvirno procjenjuje gradijent funkcije pomaka robota.

Praćenje pozicije (eng. Pose Tracking) metoda se koristi kada je poznata početna pozicija robota pa je potrebno samo pratiti njegovu poziciju kroz vrijeme. Metoda koristi ekstrakciju tj. izdvajanje značajki okoline koje se mogu uspoređivati te se tako kroz vrijeme može pratiti promjena položaja nekih uočljivih objekata.

Metoda višestrukih hipoteza (eng. Multiple Hypotesis Localization) pretpostavlja da početna pozicija nije poznata ali je poznata topografija mape. U ovome slučaju početnu poziciju može robotu pridodati korisnik ili robot uvijek može započeti iz iste pozicije. Ideja iza ove metode je da se detektira svojstvo te se preko njega stvaraju hipoteze o položaju robota naspram toga objekta kojemu pripada to svojstvo. Može se stvoriti nova hipoteza ili se može poboljšati neka od prethodnih hipoteza ili ipak eliminirati.

Metoda iteracije najbližih točaka (eng. Iterative Closest Point) minimizira razliku između dvije skupine točaka tako da iterira između svake dvije točke te pronalazi onu kombinaciju koja daje najmanju grešku. Često se koristi pri rekonstrukciji 2D ili 3D površina nakon skeniranja. Tijekom izvođenja te metode jedna skupina točaka je fiksna tj. referentna dok se druga transformira tako da se najbolje slažu koordinatama u referentnom skupu. Postoje mnoge varijante ICP-a od kojih su point-to-point (usporedba točka-točka) , point-to-plane (usporedba točke-površina) i point-to-line (usporedba točka-linija) najpopularnije.

Metoda usporedbe očitanja (eng. Scan Matching) koristi dva uzastopna očitanja senzora robota poput lasera, sonara, … da se pronađe relativan pomak robota u prostoru. Razlike između dva očitanja senzora se mogu uočiti vrlo lako zbog učestalosti skeniranja tj. frekvencije dohvaćanja senzorskih podataka te o gustoći lasera kojih uglavnom ima od nekoliko stotina do nekoliko tisuća. Načina na koji se zapravo traže razlike između dva očitanja ima mnogo. Koriste se laseri (eng. Laser Range Finders) da bi vidio prepreke i odometrija kotača (eng. Wheel Odometry) da dobije okvirno stanje robota. Odometrija iz kotača ima određenu grešku zbog proklizavanja kotača ili nekog drugog razloga te se ona tada ispravlja pomoću izračunatih vrijednosti odometrije iz lasera. U ovom radu će biti opisan te implementiran način pronalaženja odometrije pomoću korelacije histograma podataka iz lasera.

 %ok
\section{Problem lokalizacije}

Sve prethodno navedene imaju nešto zajedničko, a to je da koriste algoritme čiji rezultati nikada nisu posve točni.\\
Ta netočnost može proizaći zbog sljedećih razloga:

\begin{itemize}
  \item Šum u očitanjime - u podacima koji se dobiju iz senzora uvijek ima i podataka koji su nastali zbog privremenih objekata (npr. pas koji prolazi pokraj vozila)
  \item Sinkronizacija obrade i očitanja podataka - zbog prebrzog slanja podataka algoritmu, te se tako mogu neka očitanja preskočiti
  \item Samog načina izvedbe senzora - možda senzor zbog samog načina fizičke izvedbe ima uračunat šum
  \item \dots
\end{itemize}


Razne metode lokalizacije već unutar svog tijeka izvođenja imaju metode koje prate veličinu relativne pogreške te ju pokašaju minimizirati nakon svake iteracije ali ta pograška i dalje postoji te će uvijek i postojati. Te pogreške se robota koji rade u skladištima ne moraju uzeti previše ozbiljno, dok se kod autonomnih vozila u svakodnevnome cestovnom prometu ili industrijskih robota te pogreške moraju uvijek uzeti u obzir. %ok

\chapter{Priprema podataka}
\section{Simulator}

Za točne referentne podatke potrebno je imati simulirano okruženje. Takvo simulirano okruženje se zove simulator. Potreban je simulator koji već ima integrirane razne mape, razne senzore, vozila te način komunikacije s tim vozilima iz vanjskih skripti. Neki od simulatora su opisani u sljedećem tekstu.

\subsubsection{Carla}
\begin{figure}[ht!]
  \centering
  \includegraphics[scale=0.5]{images/carla_logo.png}
  \caption{Carla logo\cite{logo:carla}}
\end{figure}

Carla\cite{dosovitskiy17} je simulacijsko okruženje koje služi za testiranje metoda i algoritama prilikom razvoja autonomnih vozila. U pozadini koristi Unreal Engine za izvršavanje simulacije. Simulator se ponaša kao poslužitelj koji prima naredbe iz vanjskih klijentskih programa. Ti klijentski programi su pisani u programskom jeziku pytohn.
Carla ima integirane razne senzore te su neki od njih:
\begin{itemize}
  \item RGB kamera
  \item LIDAR senzor
  \item Senzor dubine
  \item GNSS
\end{itemize}

Sponzori projekta su Intel, Toyota, GM i Computer vision Center. Više o ovome simulatoru će biti u sljedećem poglavlju.

\subsubsection{Apollo}
\begin{figure}[ht!]
  \centering
  \includegraphics[scale=0.2]{images/apollo_logo.png}
  \caption{Apollo logo\cite{logo:apollo}}
\end{figure}

Apollo je također rješenje za testiranje autonomnoh vozila. Sadrži simulator ali također je i potpuno komercijalno rješenje. Podržava razne scenarije, ima sustav ocjenjivanja koji daje ocjenu na temelju desetak metrika. Simulacije zapravo provodi u oblaku tj. koristi Microsoft Azure. Sponzori projekta su mnoge azijske tvrtke kao i Ford, Microsoft, Daimler, Honda, Intel i ostali.

\subsubsection{rFpro}
\begin{figure}[ht!]
  \centering
  \includegraphics[scale=0.2]{images/rfpro_logo.jpg}
  \caption{rFpro logo\cite{logo:rfpro}}
\end{figure}

rFpro je kompletno rješenje za testiranje autonomnih vozila. U potpunosti je kommercijalno rješenje ali je zato jedno od najboljih u svijetu. Uglavnom je usredotočeno na primjenu strojnog učenja u autonomnim vozilima. Ima jednu od najvećih baza digitaliziranih stvarnih likacija diljem svijeta. Dinamički sustav vremena omogućuje testiranje ponašanja vozila u raznim vremenskim uvjetima. Sponzori projekta su BMW, Shell, GM, Renault i ostali.

\subsubsection{AVSimulation}
\begin{figure}[ht!]
  \centering
  \includegraphics{images/avsimulation_logo.png}
  \caption{AVSimulation logo\cite{logo:avsimulation}}
\end{figure}

AVSimulation se zapravo sastoji od simulatora vožnje i samog simulatora SCANeR. SCANeR je skup aplikacija koji pružaju rute, senzore, vozila, dinamičko vrijeme, pisanje skripti. Vrlo je modularan. Smulator vožnje je zapravo kupola koja se sastoji od cijelog vozila te se zapravo kretanje tog vozila simulira unutar te kupole. Sponzori su Renault, PSA, Volvo, Microsoft, Mazda i ostali.
\newpage %ok
\section{Opis okruženja}

Za okruženje se koristi Carla simulator zbog širokog programskog sučelja i otvorenosti koda, tekođer je besplatno za korištenje u sve svrhe. Simulator se pokreće kao poslužitelj te se vozila dodaju pomoću skripte koja je napisana u programskom jeziku python.

\begin{figure}[ht!]
  \centering
  \includegraphics[scale=0.5]{images/carla_town03_example.png}
  \caption{Primjer karte pod nazivom Town03}
  \label{fig:town03_exmaple}
\end{figure}

Na slici \ref{fig:town03_exmaple} se vidi izgled karte simuliranoga prostora. Korištene mape su definirane OpenDrive standardom. Simulator podržava raznolike senzore. Svi ti senzori se mogu postaviti samostalno na mapu, ali su najkorisniji kada se postave na drugo vozilo.

\newpage
\subsubsection{RGB senzor}
\begin{figure}[ht!]
  \centering
  \includegraphics[scale=0.5]{images/rgb_example.png}
  \caption{Primjer regularne kamere\cite{carla:sensors}}
  \label{fig:rgb_exmaple}
\end{figure}

RGB senzor simulira kameru koja može snimati sliku koristeću crvenu, zelenu i plavu boju, tj. regularnu kameru. Ovaj senzor može prikupljati podatke iz simulatora u video obliku ili kao niz slika. Ovaj senzor se može koristiti u metodama lokalizacije koje kao temeljni algoritam koriste prepoznavanje ostalih sudionika u prometu prema obliku tj algoritmi prepoznavaju kontekst slike.

\subsubsection{Senzor dubine}
\begin{figure}[ht!]
  \centering
  \includegraphics[scale=0.5]{images/depth_example.png}
  \caption{Primjer rezultata senzora dubine\cite{carla:sensors}}
  \label{fig:depth_exmaple}
\end{figure}

Ovaj senzor koristi nizove projicirajućih točaka da bi ilustrirao udaljenosti objekata, original na slici \ref{fig:depth_exmaple}. Tada se ti podaci pretvaraju u crno bijelu sliku gdje je svaki pixel u nijansama sive boje tj. ovisno koliko je objekt na određnome pixelu odaljen od kamere imati će svijetliju nijansu, konvertirano na slici \ref{fig:depth_exmaple}.

\newpage
\subsubsection{Semantičke segmentacije}
\begin{figure}[ht!]
  \centering
  \includegraphics[scale=0.5]{images/sem_seg_exmaple.png}
  \caption{Primjer semantičke segmentacija\cite{carla:sensors}}
  \label{fig:sem_seg_exmaple}
\end{figure}

Ovaj senzor zapravo nije senzor ali je grupiran u klasu senzora zato što radi na vrlo sličan način kao i ostali senzori u simulatoru. Ovaj senzor dijeli sliku kamere na semantičke dijelove tj. objekte različitog tipa predstavlja drugim bojama. Na slici \ref{fig:sem_seg_exmaple} se vidi da je nebo crne boje, auti su plave boje, drveće je zeleno itd. OVaj način raspoznavanja objekata je moguć samo u simulator zato što nije potrebno razpoznavati objekte na slici već su oni definirani u simuliranoj okolini.

\subsubsection{LIDAR senzor}
\begin{figure}[ht!]
  \centering
  \includegraphics[scale=0.5]{images/LIDAR_examaple.png}
  \caption{LIDAR podaci\cite{carla:sensors}}
  \label{fig:lidar_exmaple}
\end{figure}
LIDAR (eng. light detecting and raging) senzor je zapravo vertikalan skup lasera koji simuliraju skeniranje od 360 stupnjeva tako da se rotiraju određeni broj puta u sekundi. Povratni podaci senzora su zapravo točke tj. koordinate relativne naspram samoga senzora do kojih su laseri uspjeli doći. Na slici \ref{fig:lidar_exmaple} se mogu vidjeti takvi podaci vizualizirani u programu MeshLab. Više o ulaznim parametrima senzora kasnije u radu.

\subsubsection{Senzor sudara}
Ovaj senzor dojavljuje klijentskome programu ako se vozila sudarilo s drugim objektom u simulaciji.

\subsubsection{Senzor prijelaza trake}
Ovaj senzor dojavljuje klijentskome programu ako vozilo prođe preko trake na cesti.

\subsubsection{GNSS senzor}
Senzor koji dojavljuje klijentskome programu trenutnu GNSS lokaciju vozila. Ta lokacija se interno računa tako da se lokacija vozila dodaje na geografsku referentnu lokaciju definiranu za cijelu mapu.

\subsubsection{Senzor prepreke}
Ovaj senzor javlja klijentskome programu ako se ispred vozila nalazi prepreka.

\subsubsection{Promet}
Postoji poveći broj već unaprijed definiranih vozila koja se mogu koristiti. Mogu se koristiti kao nositelji senzora ili kao ostali sudionici u prometu. Carla ima dobro definirana prometna pravila te semafore da bi simulacija izgledala što vjernije.
\newpage %ok
\section{Oblak točaka}

Oblak točaka je skupina podataka koji definiraju neki objekt u prostoru. Oblak točaka se obično generira pomoću trodimenzionalnog skenera. Oblaci točaka imaju veliku primjenu u rekonstrukcijama predmeta, vizualizaciji, animaciji, virtualnoj i proširenoj stvarnosti te industrijskoj proizvodnji i kontroli kvalitete. U trodimenzionalnome kartezijevom sustavu svaka točka je definirana s tri atributa, a to su njene x, y i z koordinate. Uz te osnovne podatke svaka točka također može sadržavati i podatke o njenoj boji.

\begin{figure}[h!]
  \centering
  \includegraphics[scale=0.3]{images/point-coordinates.png}
  \caption{Ilustracija točke u kartezijevom koordinatnome prostoru}
  \label{fig:point_coordinates}
\end{figure}

\begin{figure}[h!]
  \centering
  \includegraphics[scale=0.2]{images/katedrala-point-cloud.jpg}
  \caption{Oblak točaka katedrale u Zagrebu\cite{cloud:katedrala}}
  \label{fig:point_cloud_exmaple}
\end{figure}

Na slici \ref{fig:point_cloud_exmaple} se vidi skup točaka koji opisuje katedralu u Zagrebu te okolne objekte. Skup se sastoji od oko 22 milijuna točaka. Skup točaka je mnogo jednostavnije koristiti za mapiranje objekata od slika zato što se lakše može obraditi na računalu. Ti podaci su zapravo spremljeni u tekstualne datoteke pa su lako prenosivi i čitljivi. Konkretno metode u ovome radu koriste oblak točaka prikupljen s rotirajućim laserima na vozilu. Jedan taj skup točaka predstavlja stanje okoline vozila u jednome vremenskome trenutku. %ok
\section{Referentni podaci}

Referentni podaci su oni podaci s kojima se uspoređuju rezultati metoda. Ti referentni podaci su generirani u simulatoru te predstavljaju lokaciju i rotaciju vozila u jednome trenutku. Referentni podaci se zapravo sastoje od lokacije i rotacije vozila.

\subsubsection{Lokacija}
Lokacija vozila je također definirana kao točka u kartezijevom koordinatnome prostoru. Sastoji se od x, y i z koordinata. Slično kao prikazano na slici \ref{fig:point_coordinates}.


\subsubsection{Rotacija}
 U trodimenzialnome prostoru objekt se zapravo može rotirati oko beskonaćnoga broja osi ali se u pravilu uzimaju 3 statičke osi. Te osi se nazivaju os skretanja (eng. yaw), os poniranja (eng. pitch) i os valjanja (eng. roll). 

\begin{figure}[!ht]
  \centering
  \includegraphics[scale=0.3]{images/yaw_roll_pitch_example.png}
  \caption{Ilustracija osi rotiranja}
  \label{fig:yaw_roll_pitch_example}
\end{figure}


Os rotacije je os koja prolazi u smjeru kretanja vozila (x os), os poniranja je zapravo os okomita s os rotacija (z os), dok je os skretanja okomita na obje prethodne osi (y os). Te osi su ilustrirane na slici \ref{fig:yaw_roll_pitch_example}.

Vizualizacije referentnih podataka za jedan primjer kretanja vozila možemo vidjeti na sljedećim slikama. Na svim grafovima svaka točka predstavlja jedno očitanje. Grafovi su izgrađeni pomoću programskog jezika Kotlin koristeći biblioteku XChart. Slika \ref{fig:gt_lokacija} prikazuje graf lokacija vozila. Slika \ref{fig:gt_lokacija_x} pokazuje x koordinate vozila  uvremenu. Slika \ref{fig:gt_lokacija_y} pokazuje y koordinate vozila u vremenu. Slika \ref{fig:gt_rotacija} pokazuje vrijednosti rotacija oko statičkih osi u vremenu. Za očekivati je kako će se mjenjati samo rotacija skretanja zato što vozilo može skretati, a ne može ponirati ili se valjati. 

\begin{figure}[h!]
  \includegraphics[scale=0.4]{images/gt_lokacija.png}
  \caption{Graf lokacije vozila}
  \label{fig:gt_lokacija}
\end{figure}
\pagebreak
\begin{figure}[h!]
  \includegraphics[scale=0.4]{images/gt_lokacija_x.png}
  \caption{Graf X lokacije vozila u vremenu}
  \label{fig:gt_lokacija_x}
\end{figure}
\begin{figure}[h!]
  \includegraphics[scale=0.4]{images/gt_lokacija_y.png}
  \caption{Graf Y lokacije vozila u vremenu}
  \label{fig:gt_lokacija_y}
\end{figure}
\pagebreak
\begin{figure}[h!]
  \includegraphics[scale=0.4]{images/gt_rotacija.png}
  \caption{Graf rotacija u vremenu}
  \label{fig:gt_rotacija}
\end{figure} %ok
\section{Izvor podataka}
Kao izvor podataka za algoritme se koristi već prije spomenuti LIDAR senzor. S obzirom da je senzor simuliran, možemo mu postavljati sljedeće parametre:

\begin{itemize}
  \item Broj kanala - broj lasera
  \item Donja granica polja vida - koliko nisko su orijentirani laseri
  \item Gornja granica polja vida - koliko visoko su orijentirani laseri
  \item Ukupan broj točaka - ukupan broj točaka po laseru u očitanju
  \item Frekvencija rotacije - koliko četo se laseri rotiraju
  \item Vremenski korak - koliko često se podaci očitavaju
\end{itemize}

Podaci koje nam vrati senzor su zapravo u obliku skupa točaka (eng. Point Cloud). %ok
\section{Prikupljanje podataka}

Podaci su prikupljeni tako da se simulator pokrene u poslužiteljkom načinu rada te se tada pokreće klijentska skripta napisana u python jeziku. Ta skripta uspostavi kontakt s poslušiteljem te se tako šalju naredbe. Te naredbe će zapravo stvoriti naše vozilo, ostale sudionike i senzor. Nakon što smo prikupili dovoljno podataka skripta će obrisati stvorene objekte i spremiti podatke u datoteke. Tada simulator može prekinuti s radom te se ti podaci mogu obrađivati na bilo koji način.

\subsubsection{Pokretanje simulatora}

\begin{listing}[h!]
  \begin{minted}{bash}
    CarlaUE4.exe \ 
      /Game/Carla/Maps/Town01 \
      -quality-level=Low \
      -benchmark -fps=15 \
      -windowed -ResX=800 -ResY=600 \
      -carla-port=2000 \
  \end{minted}
  \caption{Carla naredba}
  \label{coderef:carla_start}
\end{listing}

Simulator Carla se pokreće pomoću python skripte zbog boljeg upravljanja parametrima ali  se zapravo sastoji od naredbe pokazane u primjeru \ref{coderef:carla_start}. Simulacija se pokreće u mapi pod nazivom Town01. Kvaliteta je postavljena na najnižu vrijednost kao i broj slika u sekundi zbog boljih performansi izvođenja. Prozor smo postavili na vrlo malu rezoluciju od 800 pixela širine i 600 pixela visine također zbog boljih performansi. Vrlo važan parametar je sučelje preko kojega klijentski program komunicira s poslužiteljem. Ovdje je definiran kao 2000.


\subsubsection{Klijentska kripta}
Referentni i testni podaci su prikupljeni iz simualtora ali iz različitih izvora. Referentni podaci su prikupljeni iz samoga simulatora dok su testni podaci prikupljeni pomoću senzora.

\begin{listing}[h!]
  \begin{minted}[frame=lines, linenos]{python}
  class CarlaProp:
    spawn_delay = 1.0
    host = "localhost"
    port = 2000
  \end{minted}
  \caption{Carla postavke}
  \label{coderef:carla_properties}
\end{listing}

\begin{listing}[h!]
  \begin{minted}[frame=lines, linenos]{python}
  def connect_to_carla(self):
    self.client = carla.Client(
      CarlaProp.host,
      CarlaProp.port
    )
    self.client.set_timeout(2.0)
    self.world = self.client.get_world()
    settings = self.world.get_settings()
    settings.synchronous_mode = True
    self.world.apply_settings(settings)
  \end{minted}
  \caption{Uspostava konekcije s poslužiteljem}
  \label{coderef:carla_connect}
\end{listing}

U primjeru izvornoga koda \ref{coderef:carla_connect} klijent se spaja na Carla poslužitelj čiju smo lokaciju (IP adresu i sučelje) definirali u klasi \mintinline{python}{CarlaProp}. Također postavljamo sinkroni način rada simulatora, a razlog je taj što želimo upravljati frekvencijom slanja podataka iz poslužitelja prema klijentima. Varijabla \mintinline{python}{self.world} služi za izvođenje svih operacija koje su vezane uz svijet.

Svaka mapa ima već unaprijed definirane točke stvaranja tj. koordinate u svijetu na kojima možemo stvoriti objekte. Te koordinate se nalaze na cestama. Njih možemo dobiti naredbom prikazanom na primjeru izvornoga koda \ref{coderef:spawn_points}. 

\begin{listing}[h!]
  \begin{minted}[frame=lines, linenos]{python}
    def get_spawn_points(world):
      return list(world.get_map().get_spawn_points())
  \end{minted}
  \caption{Dohvaćanje liste koordinata stvaranja}
  \label{coderef:spawn_points}
\end{listing}

Sljedeće što slijedi je stvaranje ostalih sudionika prometa tj. ostalih vozila. Carla ima već unaprijed definirane nacrte raznih objekata.

\begin{listing}[h!]
  \begin{minted}[frame=lines, linenos]{python}
def get_vehicle_blueprints(world):
  blueprints = world.get_blueprint_library()
                    .filter('vehicle.*')
  blueprints = [
    x for x in blueprints
      if int(x.get_attribute('number_of_wheels')) == 4
  ]
  return [
    x for x in blueprints
      if not x.id.endswith('isetta')
  ]
  \end{minted}
  \caption{Dohvaćanje nacrta vozila}
  \label{coderef:vehicle_blueprints}
\end{listing}
\pagebreak
Na kodu \ref{coderef:vehicle_blueprints} se vidi kako koristimo knjižnicu nacrta da bi filtrirali nama potrebne nacrte. Koristiti će se samo vozila koja imaju 4 kotača.

\begin{listing}[h!]
  \begin{minted}[frame=lines, linenos, escapeinside=!!]{python}
  def spawn_npcs(self):
    blueprints = utils.get_vehicle_blueprints(self.world)
    points = self.spawn_points[1:self.npc_number+1]
    for i in range(self.npc_number):
      actor_blueprint = random.choice(blueprints)
      actor_spwn_point = points[i]
      spawned_actor = self.world.try_spawn_actor( !\label{lineref:spawn_instance}!
        actor_blueprint,
        actor_spwn_point
      )
      spawned_actor.set_autopilot() !\label{lineref:set_autopilot}!
      self.npcs.append(spawned_actor)
    self.tick()
  \end{minted}
  \caption{Stvaranje ostalih vozila}
  \label{coderef:vehicle_spawning}
\end{listing}

Koristeću točke stvaranja i nacrte vozila sada se mogu ta vozila stvoriti u svijetu. Na \ref{coderef:vehicle_spawning} se koristeći petljom stvara unaprijed zadan broj ostalih sudionika definiranih u varijabli klase \mintinline{python}{self.npc_number}. Njihove reference se tada spremaju u listu zato što se na kraju izvođenja moraju uništiti. Stvaranje instance nacrta se izvodi naredbom na liniji \ref{lineref:spawn_instance}. Također smo svakoj instanci definirali autonomni način rada na liniji \ref{lineref:set_autopilot}.

Sada se definira vozilo koje zapravo promatramo tj. koje ima na sebi lidar senzor. To se radi na približno jednak način kao u primjeru \ref{coderef:vehicle_spawning}. Izvorni kod je prikazan na primjeru \ref{coderef:actor_spawning}.


\begin{listing}[h!]
  \begin{minted}[frame=lines, linenos]{python}
  def spawn_actor(self):
    lib = self.world.get_blueprint_library()
    spawn_point = self.spawn_points[0]
    actor_blueprint = utils.get_vehicle_blueprint(lib)
    self.actor = self.world.spawn_actor(
      actor_blueprint, spawn_point
    )
    self.actor.set_autopilot()
    self.tick()
  \end{minted}
  \caption{Stvaranje promatranoga vozila}
  \label{coderef:actor_spawning}
\end{listing}

Sada slijedi pronalazak nacrta za lidar senzor, postavljanje njegovih atributa, njegovo instanciranje i postavljanje na promatrano vozilo. Izvorni kod je prikazan na \ref{coderef:lidar_find_spawn}. Postavke LIDAR senzora se nalaze u klasi \mintinline{python}{LIDARProp}.


\begin{listing}[h!]
  \begin{minted}[frame=lines, linenos]{python}
  import carla

  class LIDARProp:
    sensor_tick = str(0.0)
    channels = str(180)
    laser_range = str(1500.0)
    rotation_frequency = str(120.0)
    points_per_second = str(600_000)
    upper_fov = str(45.0)
    lower_fov = str(-80.0)
    location = carla.Transform(
      carla.Location(x=0, y=0, z=4)
    )
  \end{minted}
  \caption{LIDAR atributi}
  \label{coderef:lidar_props}
\end{listing}

Pomoću \mintinline{python}{sensor_tick} parametra se definira da simulator treba  prikupljati podatke najbrže što može. Varijabla \mintinline{python}{laser_range} definira koliko daleko laser može doprijeti, ovdje je definirano na 1500 centimetara ili 15 metara. Frekcencija rotacije lidara je definirana varijablom \mintinline{python}{rotation_frequency} i iznosi 120 rotacija u minuti. Gornja granica mjerenja lasera je 45°, a donja je -80°. parametar \mintinline{python}{location} definira lokaciju lasera tj. bit će u središtu koordinatnog sustava ali na visini od 4 metra. Sredina tog koordinatnog sustava je zapravo sredina unutarnjeg sustava vozila na kojemu će taj senzor biti pričvršćen.

\pagebreak
\begin{listing}[h!]
  \begin{minted}[frame=lines, linenos, escapeinside=!!]{python}
    def connect_LIDAR(self):
      lib = self.world.get_blueprint_library()
      blueprint = utils.get_lidar_sensor_blueprint(lib) !\label{lineref:get_lidar_blueprint}!
      blueprint.set_attribute( !\label{lineref:lidar_prop_start}!
        'sensor_tick',
        LIDARProp.sensor_tick
      )
      blueprint.set_attribute(
        'channels',
        LIDARProp.channels
      )
      blueprint.set_attribute(
        'range', LIDARProp.laser_range)
      blueprint.set_attribute(
        'rotation_frequency',
        LIDARProp.rotation_frequency
        )
      blueprint.set_attribute(
        'points_per_second',
        LIDARProp.points_per_second
        )
      blueprint.set_attribute(
        'upper_fov',
        LIDARProp.upper_fov
        )
      blueprint.set_attribute(
        'lower_fov',
        LIDARProp.lower_fov
      ) !\label{lineref:lidar_prop_end}!
      utils.print_sensor_blueprint_data(blueprint)
      self.lidar = self.world.try_spawn_actor(  !\label{lineref:connect_lidar}!
        blueprint, LIDARProp.lidar_relative_postion,
        attach_to=self.actor
      )
      self.lidar.listen(  !\label{lineref:lidar_callback_set}!
        lambda data: self.lidar_callback(data)
      )
      self.tick()
  \end{minted}
  \caption{Stvaranje LIDAR senzora}
  \label{coderef:lidar_find_spawn}
\end{listing}

Na liniji \ref{lineref:get_lidar_blueprint} primjera \ref{coderef:lidar_find_spawn} dohvaćamo nacrt lidar senzora. Tada od linije \ref{lineref:lidar_prop_start} do \ref{lineref:lidar_prop_end} postavljamo zadane postavke nad nacrtom. Konačno na liniji \ref{lineref:connect_lidar} stvaramo instancu senzora ali metodi predajemo dodatan parametar \mintinline{python}{attach_to} koji je je jednak referenci na naše vozilo. Također umjesto stvarnih koordinata, za lokaciju senzora postavljamo lokaciju relativnu naspram lokacije vozila. Na liniji \ref{lineref:lidar_callback_set} postaljamo metodu \mintinline{python}{lidar_callback()} kao metodu koju će simulator pozvati svaki puta kada senzor očita okolinu i pošalje podatke.
Spremanje podataka se izvršava tek nakon što smo sakupili konačan broj očitanja. Za spremanje podataka u datoteke se koristi posebna klasa \mintinline{python}{DataSaver} pokazana na primjeru \ref{coderef:data_saver}.

\begin{listing}[h!]
  \begin{minted}[frame=lines, linenos, escapeinside=!!]{python}
import utils as utils
import time
import threading
from concurrent.futures import
  ThreadPoolExecutor, as_completed
from lidar_properties import
  LIDARProperties

class DataSaver:

  def __init__(self):
    self.pc_path = 'output/pointclouds'
    self.trans_path = 'output/actortransforms'
    self.rel_path = 'output/relative_transform.txt'

  def initialize_folders(self):
    utils.create_directory(self.trans_path)
    utils.create_directory(self.pc_path)

  def save(self, scans):
    self.initialize_folders()
    self.save_rel_trans()
    start_time = time.time()

    with ThreadPoolExecutor(max_workers=10) as executor:
      jobs = list()
      for i, (scan, transform) in enumerate(scans[1:]):
        job = executor.submit(
          self.process_pair, scan, transform, i
        )
        jobs.append(job)
      for future in as_completed(jobs):
        future.result()

  \end{minted}
  \caption{Klasa za spremanje podataka}
  \label{coderef:data_saver}
\end{listing}

\begin{listing}[h!]
  \begin{minted}[frame=lines, linenos, escapeinside=!!]{python}
  def process_pair(self, scan, trans, i):
    frame_number = scan.frame_number
    s_path = f'{self.pc_path}/{frame_number:06d}.ply'
    self.save_scan(scan, s_path, i + 1)
    tpath =  f'{self.trans_path}/{frame_number:06d}.txt'
    self.save_transform(
      trans, scan.timestamp, tpath, i + 1
    )

  def save_rel_trans(self):
    with open(self.rel_path, 'w+') as file:
      file.write(
        utils.transform_to_string(LIDARProp.location)
      )

  def save_scan(self, scan, path, i = 0):
      scan.save_to_disk(path)

  def save_transform(self, trans, timestamp, path, i = 0):
    with open(path, 'w+') as file:
      file.write(
        f"{timestamp}\n{utils.transform_to_string(trans)}"
      )
  \end{minted}
  \caption{Klasa za spremanje podataka - nastavak}
  \label{coderef:data_saver_cont}
\end{listing}

Primjeri \ref{coderef:data_saver} i \ref{coderef:data_saver_cont} prikazuju klasu \mintinline{python}{DataSaver} koja služi za spremanje podataka u datoteke. Datoteke s podacima iz senzora se nalaze u direktoriju s nazivom pointclouds, dok se datoteke s podacima o lokaciji vozila nalaze u direktoriju s nazivom actortransforms. Oba ta direktorija se nalaze u direktoriju s nazivom $output$. Datoteke s informacijama o lokaciji vozila se sastoje od 2 reda. Prvi red sadrži vremensku oznaku a drugi sadrži lokaciju i transformaciju koji su opisani u prijašnjem poglavlju. Datoteke koje sadrže podatke o jednome očitanju se nalaze u tekstualnim datotekama s ekstenzijom .ply te se sastoji od zaglavlja i po jedan redak za svaku točku u očitanju. Također postoji još jedna datoteka koja samo sadrži relativnu transformaciju između senzora i vozila. Ona se nalazi u direktoriju $output$.

\begin{listing}[h!]
  \begin{minted}[frame=lines]{html}
    ply
    format ascii 1.0
    element vertex 17252
    property float32 x
    property float32 y
    property float32 z
    end_header
    -6.6331 4.7159 -3.6613
    -7.6668 3.6868 -3.6721
    -6.8233 4.8510 -3.6132
    -7.2357 4.3627 -3.6467
    -7.5643 3.8149 -3.6567
    -6.8332 4.8581 -3.5678
    .
    .
    .
  \end{minted}
  \caption{Izgled sadržaja .ply datoteke}
  \label{files:ply_format}
\end{listing} %ok

\chapter{Algoritmi lokalizacije}
\section{Obitelj algoritama}

Koruišteni algoritmi su ... %ok
\section{Dijeljeni kod i ulazni podaci}

Kod u nastavku je zajednički za oba algoritma tj. nema veze s samim algoritmom već služi za dohvat podataka i spremanje rezultata algoritma. Sastoji se od metoda za čitanje datoteka s informacijama o oblacima točaka i metoda za spremanje transformacijskih matrica u datoteke. 

\begin{listing}[h!]
  \begin{minted}[frame=lines, linenos]{text}
typedef PointXYZ PT;
typedef PointCloud<PT> PointCloudType;
typedef IterativeClosestPoint<PT, PT, double> ICP;

PointCloudType::Ptr cloud_ref(new PointCloudType());
PointCloudType::Ptr cloud_target(new PointCloudType());
PointCloudType::Ptr cloud_reg(new PointCloudType());

PointCloudType::Ptr cloud_ref_f(new PointCloudType());
PointCloudType::Ptr cloud_target_f(new PointCloudType());

string root_point_clouds = "\\point_clouds\\";
string root_results = "\\icp_results\\";
  \end{minted}
  \caption{Generalizirani ICP - konstante}
  \label{coderef:gen_icp_const}
\end{listing}

U primjeru izvornoga koda \ref{coderef:gen_icp_const} su definirane konstante poput putanje za spremanje rezultata i putanje s ulaznim datotekama. Također su definirani tipovi točaka \mintinline{text}{PT} kao \mintinline{text}{PointXYZ} koje će algoritam koristit te sadrže samo x, y i z koordinate. Mogu se koristiti i drugi oblici točaka. Oblak točaka \mintinline{text}{PointCloudType} je definiran kao kolekcija prethodno definiranih točaka. Naposljetku se definira tip \mintinline{text}{ICP} algoritma tj. s kojim tipovima podataka radi. Definiran je pomoću uređene trojke \mintinline{text}{<PT, PT, double>} što znači da uspoređuje točke tipa \mintinline{text}{PT}, a rezultate u transformacijsku matricu zapisuje kao \mintinline{text}{double} vrijednosti tj. kao brojeve s pomičnim zarezom dvostruke preciznosti.

Definirane su i varijable \mintinline{text}{cloud_ref} koja pokazuje na referentni skup točaka, \mintinline{text}{cloud_target} koja pokazuje na ciljni skup točaka i \mintinline{text}{cloud_reg} koja pokazuje na aproksimiran skup točaka nakon poravnanja. Za algoritme koji koriste filrirane oblake točaka koriste se varijable \mintinline{text}{cloud_ref_f} i \mintinline{text}{cloud_target_f}. One su tipa \mintinline{text}{boost::shared_ptr} te se kao takve predaju metodama kao pokazivači.

\begin{listing}[h!]
  \begin{minted}[frame=lines, linenos]{text}
ICP setupICP() {
 ICP icp;
 icp.setMaxCorrespondenceDistance(0.05);
 icp.setMaximumIterations(500);
 icp.setTransformationEpsilon(1e-8);
 icp.setEuclideanFitnessEpsilon(1);
 return icp;
}
  \end{minted}
  \caption{Instance ICP algoritma}
  \label{coderef:gen_icp_def}
\end{listing}
\pagebreak
U primjeru \ref{coderef:gen_icp_def} se definira ICP algoritam tako da mu se predaju uvjeti zaustavljanja te ostali parametri. Trenutno su zadana tri uvjeta zaustavljanja, a oni su:

\begin{enumerate}
  \item \mintinline{text}{setMaxCorrespondenceDistance} - uzima u obzir samo točke unutar zadanoga promjera u metrima
  \item \mintinline{text}{setMaximumIterations} - maksimalan broj iteracija prilikom estimacije tarnsformacije za neku točku
  \item \mintinline{text}{setTransformationEpsilon} - maksimalna dozvoljena pogreška
\end{enumerate}

\begin{listing}[h!]
  \begin{minted}[frame=lines, linenos]{text}
vector<path> get_files() {
 vector<path> paths;
 path p(root_point_clouds);
 directory_iterator end_itr;
 for (directory_iterator itr(p); itr != end_itr; ++itr) {
  if (is_regular_file(itr->path())) {
   paths.push_back(itr->path());
  }
 }
 return paths;
}
  \end{minted}
  \caption{Skupljanje ulaznih datoteka}
  \label{coderef:gen_icp_collect_files}
\end{listing}
\pagebreak
Kod u primjeru \ref{coderef:gen_icp_collect_files} koristi metode biblioteke Boost\cite{boost} za iteriranje datoteka sa skupovima točaka te vrača vektora s njihovim putanjama. Kod za stvaranje grafova je jednak bez obzira na korišteni algoritam. Grafovi su stvoreni pomoću jezika Kotlin i biblioteke XCharts. Podaci koji vizualiziraju su usporedbe stvarnih podataka tj. referentnih i podataka dobivenih pomoću algoritama. S obzirom da nam algoritmi kao izlaz daju samo transformacijske matrice, potrebno je nekako te matrice prikazati kao koordinate lokacija i kuteve rotacija.

\begin{listing}[h!]
  \begin{minted}[frame=lines, linenos]{text}
fun calculatePointsExperimental(
 icp: List<TransformMatrix>,
 first: Transform
): List<Array<DoubleArray>> {
   val transformations = mutableListOf(pointToMat(first))
   val last = transformations.last()
   icp.forEach { transform ->
    val result = mult(last, transform.matrix)
    transformations.add(result)
   }
  return transformations
}
  \end{minted}
  \caption{Generiranje estimiranih transformacijskih matrica}
  \label{kotlin:gen_est_loc}
\end{listing}

\begin{listing}[h!]
  \begin{minted}[frame=lines, linenos, escapeinside=!!]{text}
fun pointToMat(first:  Transform): Array<DoubleArray> {
 val rot = first.rotation
 val loc = first.location
 val rx = rot.toRx()
 val ry = rot.toRy()
 val rz = rot.toRz()
 val res = multi(rz, multi(ry, rx)) !\label{lineref:multiply_matrices_xyz}!
 return arrayOf(
  arrayOf(res[0][0], res[0][1], res[0][2], loc.x),
  arrayOf(res[1][0], res[1][1], res[1][2], loc.y),
  arrayOf(res[2][0], res[2][1], res[2][2], loc.z),
  arrayOf(0.0, 0.0, 0.0, 1.0)
 )
}
  \end{minted}
  \caption{Generiranje estimiranih lokacija}
  \label{kotlin:init_mat_method}
\end{listing}

Kod u primjeru \ref{kotlin:gen_est_loc} se vidi metoda za generiranje estimiranih točaka iz prve referentne točke i transformacijskih matrica. Kao argumente metoda prima listu transformacijskih matrica \mintinline{text}{icp} te transformaciju prve referentne točke. Algoritam radi tako da prvo izračuna transforamcijsku matricu iz prve referentne točke pomoću metode \ref{kotlin:init_mat_method}. Zapravo se rotacija iz Eulerovog oblika pretvara u zasebne rotacijske matrice te njih pomnoži redosljedom XYZ zbog redosljeda rotacija prilikom ICP estimacije transformacije oblaka točaka. Izgledi pojedinih matrica su prikazani u poglavlju Referentni podaci. Tada se vraća matricu veličine 4x4 koja se sastoji od rotacijske matrice i lokacija. Algoritam koristi prethodno estimiranu transformacijsku matricu te trenutnu estimiranu transformacijsku matricu tako da iz pomnoži pravilima množenja matrica. Tada se dobije nova matrica koja zapravo predstavlja nove koordinate i nove rotacije naspram prethodne. Na liniji \ref{lineref:multiply_matrices_xyz} primjera \ref{kotlin:gen_est_loc} se vidi kako se zapravo primjenjuje formula $R = R_{z}(\gamma)R_{y}(\beta)R_{x}(\alpha)$. Taj redosljed množenja matrica rotacija je vrlo bitan.

Vizualizacije referentnih podataka za oba primjera kretanja vozila možemo vidjeti na sljedećim grafovima. Na njima svaka točka predstavlja jedno očitanje. Slika \ref{fig:gt1_lokacija} prikazuje graf lokacija vozila. Slika \ref{fig:gt1_lokacija_koord} pokazuje koordinate vozila u vremenu. Možemo vidjeti da se mijenjaju samo x i y koordinate zato što vozilo može samo skretati. Slika \ref{fig:gt1_rot_vr} prikazuje valjanje, poniranje i skretanje oko statičnih osi vozila u vremenu. Također se može uoćiti da se samo skretanje mijenja zato što se vozilo ne može valjati niti ponirati. Slika \ref{fig:gt1_rot_kv} također pokazuje rotaciju vozila u vremenu ali predstavljenu u obliku kvaterniona.
\pagebreak
\subsubsection{Primjer referentnih podataka A}
Prvi skup podataka se sastoji od 300 očitanja tj. postoji 300 datoteka sa skupovima točaka. Duljina trajanja te simulacije je 200 sekundi. Parametri koji su korišteni u python skripti za postavljanje lidar senzora su sljedeći:
\begin{enumerate}
  \item Maksimalna udaljenost laserske zrake je postavljena na 1500 cm tj. 15 metara
  \item Maksimalan skup točaka u jednome očitanju je postavljen na 600 000.
\end{enumerate}
\begin{figure}[H]
  \includegraphics[scale=0.4]{images/koordinate1.png}
  \caption{Graf lokacije vozila}
  \label{fig:gt1_lokacija}
\end{figure}
\begin{figure}[H]
  \includegraphics[scale=0.4]{images/koordinate_vrijeme1.png}
  \caption{Graf x, y i z koordinata vozila u vremenu}
  \label{fig:gt1_lokacija_koord}
\end{figure}
\begin{figure}[H]
  \includegraphics[scale=0.4]{images/rotacija_vrijeme1.png}
  \caption{Valjanje, skretanje i poniranje vozila u vremenu}
  \label{fig:gt1_rot_vr}
\end{figure}
\begin{figure}[H]
  \includegraphics[scale=0.4]{images/rotacija_kvaterni1.png}
  \caption{Rotacija vozila u obliku kvaterniona}
  \label{fig:gt1_rot_kv}
\end{figure}

\newpage
\subsubsection{Primjer referentnih podataka B}
Drugi skup podataka se sastoji od 600 očitanja tj. postoji 600 datoteka sa skupovima točaka. Duljina trajanja te simulacije je 80 sekundi. Parametri koji su korišteni u python skripti za postavljanje lidar senzora su sljedeći:
\begin{enumerate}
  \item Maksimalna udaljenost laserske zrake je postavljena na 2000 cm tj. 20 metara
  \item Maksimalan skup točaka u jednome očitanju je postavljen na 1 000 000.
\end{enumerate}
\begin{figure}[H]
  \includegraphics[scale=0.35]{images/koordinate2.png}
  \caption{Graf lokacije vozila}
  \label{fig:gt2_lokacija}
\end{figure}
\begin{figure}[H]
  \includegraphics[scale=0.35]{images/koordinate_vrijeme2.png}
  \caption{Graf x, y i z koordinata vozila u vremenu}
  \label{fig:gt2_lokacija_koord}
\end{figure}
\begin{figure}[H]
  \includegraphics[scale=0.35]{images/rotacija_vrijeme2.png}
  \caption{Valjanje, skretanje i poniranje vozila u vremenu}
  \label{fig:gt2_rot_vr}
\end{figure}
\begin{figure}[H]
  \includegraphics[scale=0.35]{images/rotacija_kvaterni2.png}
  \caption{Rotacija vozila u obliku kvaterniona}
  \label{fig:gt2_rot_kv}
\end{figure}
Prvi niz skupova točaka se skupljao u dužem periodu ali ima samo 300 skupova zato što se koristi svako 15 očitanje. Drugi niz skupova točaka ima 600 očitanja zato što se uzima svako očitanje. Također je drugi skup detaljniji od prvoga što pridonosi točnosti algoritama. Duljinu puta možemo izračunati tako da pronađemo udaljenosti između dvije sljedne točke te ih zbrojimo. Tako dobivena udaljenost za referentni primjer A iznosi 662.3 metra, a za referentni primjer B iznosi 448.2 metra.

\begin{listing}[H]
  \begin{minted}[frame=lines, linenos]{text}
void save_matrix(ICP icp, string first, string second) {
 Matrix4d transformation = icp.getFinalTransformation();
 double fitness = icp.getFitnessScore();
 string filename = first + "-" + second + ".txt";
 Matrix3d mat = mat4x4_to_3x3(transformation);
 Vector3d rpy = mat.eulerAngles(0, 1, 2);
 save_to_file(
   filename,
   mat_to_string(transformation),
   fitness,
   rpy
  );
}
  \end{minted}
  \caption{Generalizirani ICP - spremanje rezultata}
  \label{coderef:gen_icp_save_matrix}
\end{listing}

Spremanje rezultata se vrši metodom \mintinline{text}{save_matrix} u primjeru \ref{coderef:gen_icp_save_matrix}. Matricu transformacije možemo dobiti pozivom \mintinline{text}{icp.getFinalTransformation()} te je  oblika \mintinline{text}{Matrix4d} tj. ima 4 redaka i 4 stupaca dok su elemnti tipa \mintinline{text}{double}. Ta matrica se tada transformira u matricu veličine 3x3 tj. izdvaja se rotacijska matrica zato što takav tip matrice ima ugrađenu metodu \mintinline{text}{eulerAngles()}. Ta metoda kao argumente prima redosljed rotacija objekta tj. redosljed osi rotacija. U ovome slučaju se prvo predaje 0 što znači da se objekt prvo rotirao oko x osi, tada se predaje 1 što znači da se tada rotirao oko y osi i naposljetku se predaje 2 što znači da je zadnja rotacija bila oko z osi. Metoda vraća vektor od tri elementa koji predstavljaju valjanje, poniranje i skretanje. Konačno se sve te informacije spremaju u datoteku s imenom sastavljenim od dva indetifikatora očitanja tako da se zna koji skupovi točaka su upoređivani. Struktura te datoteke je prikazana na primjeru  \ref{files:icp_file_result}. Prva linija sadrži fitnes vrijednost. Od druge do pete linije se nalazi transformacijska matrica dok se na zadnjoj liniji nalaze Eulerovi kutevi u radijanima.
\begin{listing}[H]
  \begin{minted}[frame=lines]{text}
0.0263544
    0.999735   -0.0230172  -0.00046344  -0.00394583
   0.0230173     0.999735  0.000237121  0.000806379
 0.000457859 -0.000247725            1    0.0091155
           0            0            0            1
3.14136 -3.14113 -3.11857
  \end{minted}
  \caption{ICP - datoteka s rezultatom}
  \label{files:icp_file_result}
\end{listing}

\begin{equation}
  \begin{aligned}
AM_{x} &= \frac{1}{n}\sum_{i=0}^{n-1} |X_{i} - \hat{X}_{i}|\\
AM_{y} &= \frac{1}{n}\sum_{i=0}^{n-1} |Y_{i} - \hat{Y}_{i}|\\
AM_{z} &= \frac{1}{n}\sum_{i=0}^{n-1} |Z_{i} - \hat{Z}_{i}|
  \end{aligned}
  \label{eq:coord_am}
\end{equation}
\begin{equation}
  \begin{aligned}
MSE_{x} &= \frac{1}{n}\sum_{i=0}^{n-1} |X_{i} - \hat{X}_{i}|^2\\
MSE_{y} &= \frac{1}{n}\sum_{i=0}^{n-1} |Y_{i} - \hat{Y}_{i}|^2\\
MSE_{z} &= \frac{1}{n}\sum_{i=0}^{n-1} |Z_{i} - \hat{Z}_{i}|^2
  \end{aligned}
  \label{eq:coord_mse}
\end{equation}

\begin{equation}
  \begin{aligned}
ROT_{r} &= \frac{1}{n}\sum_{i=0}^{n-1} |R_{i} - \hat{R}_{i}|\\
ROT_{p} &= \frac{1}{n}\sum_{i=0}^{n-1} |P_{i} - \hat{P}_{i}|\\
ROT_{y} &= \frac{1}{n}\sum_{i=0}^{n-1} |Y_{i} - \hat{Y}_{i}|
  \end{aligned}
  \label{eq:rot_am}
\end{equation}
\begin{equation}
  \begin{aligned}
ROT_{r} &= \frac{1}{n}\sum_{i=0}^{n-1} |R_{i} - \hat{R}_{i}|^2\\
ROT_{p} &= \frac{1}{n}\sum_{i=0}^{n-1} |P_{i} - \hat{P}_{i}|^2\\
ROT_{y} &= \frac{1}{n}\sum_{i=0}^{n-1} |Y_{i} - \hat{Y}_{i}|^2
  \end{aligned}
  \label{eq:rot_mse}
\end{equation}


Konkretnije numeričke estimacije i pogreške se računaju pomoću sljedećih formula. Sretnja aritmetička vrijednosti razlika koordinata i rotacija se računa s formulama \ref{eq:coord_am} i \ref{eq:rot_am}. Element $n$ je ukupan broj točaka, $X_{i}$ je stvarna vrijednost koordinate x, dok je $\hat{X}_{i}$ estimirani iznos x koordinate. Po istome principu vrijedi za x i y koordinate. Element $R_{i}$ je stvarna vrijednost kuta valjanja, dok je $\hat{R}_{}$ estimirana vrijednost kuta valjanja. Po istome principu vrijedi za skretanje i poniranje. Srednje kvadratične pogreška koordinata i rotacije se računaju s formulama \ref{eq:coord_mse} i \ref{eq:rot_mse}. AM znači aritmetička sredina razlika, dok MSE znači srednja kvadratična pogreška. Prilikom računanja srednjih pogrešaka kuteva, a tako i razlike kuteva, uzeta je periodičnost kuteva u obzir.
\pagebreak %ok

\chapter{Algoritmi}
\subsection{Generalizirani ICP algoritam}

\subsubsection{Opis algoritma}
Ovaj način uspoređivanja skupova točaka je najjednostavniji. Ne tražimo prepoznatljive oblike niti imamo ikakve posebne optimizacije. Kao ulaz koristimo niz .ply datoteka. Svaka ta dototeka predstavlja jedan skup točaka tj. jedno očitanje lidar-a. Oblik datoteke je prikazan na slici \ref{files:ply_format}. Program otvari dvije datoteke koje predstavljaju dva sljedna očitanja. Tada njihov sadržaj preda metodi koja vraća transformacijsku matricu i fitnes veličinu. Za rad s datotekama se koristi Boost biblioteka otvorenoga koda.


\subsubsection{Izvorni kod algoritma}

\begin{listing}[h!]
  \begin{minted}[frame=lines, linenos]{c++}
typedef PointXYZ PT;
typedef PointCloud<PT> PointCloudType;
typedef IterativeClosestPoint<PT, PT, double> ICP;

PointCloudType::Ptr cloud_ref(new PointCloudType());
PointCloudType::Ptr cloud_target(new PointCloudType());
PointCloudType::Ptr cloud_reg(new PointCloudType());

string root_point_clouds = "\\point_clouds\\";
string root_results = "\\icp_results\\";
  \end{minted}
  \caption{Generalizirani ICP - konstante}
  \label{coderef:gen_icp_const}
\end{listing}

U primjeru izvornoga koda \ref{coderef:gen_icp_const} su definirane konstante poput putanje za spremanje rezultata i putanje s ulaznim datotekama. Također su definirani tipovi točaka \mintinline{c++}{PT} kao \mintinline{c++}{PointXYZ} koje će algoritam koristit te sadrže samo x, y i z koordinate. Mogu se koristiti i drugi oblici točaka. Oblak točaka \mintinline{c++}{PointCloudType} je definiran pomoću prethodne definicije točke. Naposljetku se definira tip \mintinline{c++}{ICP} algoritma tj. s kojim timovima podataka radi. Definiran je pomoću uređene trojke \mintinline{c++}{<PT, PT, double>} što znači da uspoređuje točke tipa \mintinline{c++}{PT}, a rezultate u transformacijsku matricu zapisuje kao \mintinline{c++}{double} vrijednosti.

Definirane su i varijable \mintinline{c++}{cloud_ref} koja pokazuje na referentni skup točaka, \mintinline{c++}{cloud_target} koja pokazuje na ciljni skup točakai \mintinline{c++}{cloud_reg} koja pokazuje na skup točaka nakon poravnanja. One su tipa \mintinline{c++}{boost::shared_ptr} te se kao takve predaju metodama kao pokazivači.

\begin{listing}[h!]
  \begin{minted}[frame=lines, linenos]{c++}
ICP setupICP() {
 ICP icp;
 icp.setMaxCorrespondenceDistance(0.05);
 icp.setMaximumIterations(500);
 icp.setTransformationEpsilon(1e-8);
 icp.setEuclideanFitnessEpsilon(1);
 return icp;
}
  \end{minted}
  \caption{Generalizirani ICP - definicija ICP}
  \label{coderef:gen_icp_def}
\end{listing}

U primjeru \ref{coderef:gen_icp_def} se definira ICP algoritam tako da mu se predaju uvjeti zaustavljanja te ostali parametri. Trenutno su zadana tri uvjeta zaustavljanja, a oni su:

\begin{enumerate}
  \item \mintinline{c++}{setMaxCorrespondenceDistance} - uzima u obzir samo točke unutar zadanoga promjera u metrima
  \item \mintinline{c++}{setMaximumIterations} - maksimalan broj iteracija prilikom estimacije matrice za neku točku
  \item \mintinline{c++}{setTransformationEpsilon} - maksimalna dozvoljena pogreška
\end{enumerate}

\begin{listing}[h!]
  \begin{minted}[frame=lines, linenos]{c++}
vector<path> get_files() {
 vector<path> paths;
 path p(root_point_clouds);
 directory_iterator end_itr;
 for (directory_iterator itr(p); itr != end_itr; ++itr) {
  if (is_regular_file(itr->path())) {
   paths.push_back(itr->path());
  }
 }
 return paths;
}
  \end{minted}
  \caption{Generalizirani ICP - skupljanje datoteka}
  \label{coderef:gen_icp_collect_files}
\end{listing}

Kod u primjeru \ref{coderef:gen_icp_collect_files} koristi metode biblioteke Boost za iteriranje datoteka sa skupovima točaka te vrača vektora s njihovim apsolutnim putanjama.

\begin{listing}[h!]
  \begin{minted}[frame=lines, linenos]{c++}
void load_point_cloud(string path, PointCloudType& cloud) {
 pcl::io::loadPLYFile(path, cloud);
}

void process_files(vector<path> paths, ICP icp) {
 for (long i = 0; i < paths.size() - 1; i++) {
  load_point_cloud(paths.at(i).string(), *cloud_ref);
  load_point_cloud(paths.at(i + 1).string(), *cloud_target);
  string first = paths.at(i).stem().string();
  string second = paths.at(i + 1).stem().string();
  icp.setInputCloud(cloud_ref);
  icp.setInputTarget(cloud_target);
  icp.align(*cloud_reg);
  if (icp.hasConverged()) {
   save_matrix(icp, first, second);
  }
  *cloud_ref = *cloud_target;
 }
}
  \end{minted}
  \caption{Generalizirani ICP - procesiranje datoteka}
  \label{coderef:gen_icp_process_load}
\end{listing}

Metodom \mintinline{c++}{process_files} u primjeru \ref{coderef:gen_icp_process_load} se iterira kroz datoteke te se otvaraju u parovima i njihov sadržaj tj. informacije o oblaku točaka se spremaju u globalne varijable \mintinline{c++}{cloud_ref} i \mintinline{c++}{cloud_target}. Na linijama 11 i 12 postavljam ICP algoritmu dodatne ulazne parametre, a to su te varijable. Konačno se pokreće algoritam te se ispituje ako je došlo do konvergencije. Do konvergencije dolazi ako su dva skupa oblaka slična tj. ako predstavljaju isti objekt ili scenu. Očekuje se da uvijek dođe do konvergencije u ovome primjeru. Ako je došlo do konvergencije, spremamo podatke u datoteku. Naposljetku se iz optimizacijskih razloga vrijednost matrice \mintinline{c++}{cloud_target} sprema

\subsubsection{Rezultat algoritma}
\begin{listing}[h!]
  \begin{minted}[frame=lines, linenos]{c++}
void save_matrix(ICP icp, string first, string second) {
 Eigen::Matrix4d transformation = icp.getFinalTransformation();
 double fitness = icp.getFitnessScore();
 string filename = first + "-" + second + ".txt";
 Eigen::Matrix3d mat = mat4x4_to_3x3(transformation);
 Eigen::Vector3d rpy = mat.eulerAngles(0, 1, 2);
 save_to_file(filename, mat_to_string(transformation), fitness, rpy);
}
  \end{minted}
  \caption{Generalizirani ICP - spremanje matrice}
  \label{coderef:gen_icp_save_matrix}
\end{listing}

Spremanje rezultata se vrši metodom \mintinline{c++}{save_matrix} u primjeru \ref{coderef:gen_icp_save_matrix}. Matricu transformacije možemo dobiti pozivom \mintinline{c++}{icp.getFinalTransformation()} te je ona oblika \mintinline{c++}{Eigen::Matrix4d} tj. ima 4 redaka i 4 stupaca dok su elemnti tipa \mintinline{c++}{double}. Ta matrica se tada transformira u matricu veličine 3x3 tj. izdvaja se rotacijska matrica zato što takav tip matrice ima ugrađenu metodu \mintinline{c++}{eulerAngles()}. Ta metoda kao argumente prima redosljed rotacija objekta tj. redosljed osi rotacija. U ovome slučaju se prvo predaje 0 što znači da se objekt prvo rotirao oko x osi, tada se predaje što znači da se tada rotirao oko y osi i naposljetku se predaje 2 što znači da je zadnja rotacija bila oko z osi. Metoda vraća vektor od tri elementa koji predstavljaju valjanje, poniranje i skretanje. Konačno se sve te informacije spremaju u datoteku s imenom sastavljenim od dva indetifikatora očitanja tako da se zna koji skupovi točaka su upoređivani. Struktura te datoteke je prikazana na primjeru  \ref{files:icp_file_result}. Prva linija sadrži fitnes vrijednost. Od druge do pete linije se nalazi transformacijska matrica dok se na zadnjoj liniji nalaze Eulerovi kutevi u radijanima.
\begin{listing}[h!]
  \begin{minted}{html}
0.0263544
    0.999735   -0.0230172  -0.00046344  -0.00394583
   0.0230173     0.999735  0.000237121  0.000806379
 0.000457859 -0.000247725            1    0.0091155
           0            0            0            1
3.14136 -3.14113 -3.11857
  \end{minted}
  \caption{ICP - datoteka s rezultatom}
  \label{files:icp_file_result}
\end{listing}

\subsubsection{Evaluacija rezultata}
opis konkretnih ulaza podataka:
  broj datoteka
  duljina trajanja simulacije

prikazani na grafovima s referentnim podacima
kotlin kod za generiranje estimiranih točaka
grafovi za :
  lokaciju,
  euler kuteve,
  razlike x,y,z
  razlike eulerovih kuteva
ispiši stvarnu duljinu putovanja
ispiši estimiranu duljinu putovanja
ispiši MEA %ok
\section{ICP algoritam s grupiranjem točaka}

\subsubsection{Opis algoritma}
Koristi se metoda iteracije najbližih točaka s grupiranjem točaka\cite{pcl:voxelgrid}. Radi tako da se cijeli oblak točaka podijeli na kocke zadane veličine te se tada unutar te kocke filtriraju točke na temelju njihove centroide\cite{wiki:Centroid}. U jednome od očitanja iz primjera A se broj točaka s 35222 smanjio na 1548 što značajno ubrzava ICP algoritam. Ovaj algoritam zapravo aproksimira skup točaka samo jednom točkom. U primjeru \ref{coderef:voxel_grouping} se vidi da je parametar visine, širine i duljine kocke uvijek jednak i iznosi 1 centimetar što znači da će se oblak točaka podijeliti na kocke veličine 1 centimetar. Performanse toga algoritma kao i točnost reprezentacije originalnog oblaka ovise o tom parametru kao i gustoći točaka.


\subsubsection{Izvorni kod algoritma}
Kod je vrlo sličan kao i u prethodnome algoritmu samo što sada imamo jedan korak više prije obrade oblaka ICP algoritmom. Doajemo linije \ref{lineref:voxel1} i \ref{lineref:voxel2} s pozivima metode \ref{coderef:voxel_method}. Unutar nje se inicijalizira objekt tipa \mintinline{text}{VoxelGrid<PT>}. Kao parametre prima oblak točaka i veličinu područja na koje će podijeliti oblak točaka.

\begin{listing}[H]
  \begin{minted}[frame=lines, linenos]{text}
void downsample_using_voxel_grid(
  PointCloudType::Ptr& cloud,
  float width, float height, float length,
  PointCloudType::Ptr& downsampled
 ) {
  VoxelGrid<PT> vg;
  vg.setInputCloud(cloud);
  vg.setLeafSize(width, height, length);
  vg.filter(*downsampled);
}
  \end{minted}
  \caption{Metoda za grupaciju točaka}
  \label{coderef:voxel_method}
\end{listing}

\begin{listing}[H]
  \begin{minted}[frame=lines, linenos, escapeinside=!!]{text}
void process_files(vector<path> paths, ICP icp) {
 for (int i = 0; i < paths.size() - 1; i++) {
  if (i == 0) {
   load_point_cloud(paths.at(i).string(), *cloud_ref);
  }
  load_point_cloud(paths.at(i + 1).string(), *cloud_target);

  string first = paths.at(i).stem().string(); 
  string second = paths.at(i + 1).stem().string();

  if (i == 0) {
   downsample_using_voxel_grid(
     cloud_ref, 1.0f, cloud_ref_filtered !\label{lineref:voxel1}!
   );
  }
  downsample_using_voxel_grid(
    cloud_target, 1.0f, cloud_target_filtered !\label{lineref:voxel2}!
  );

  icp.setInputCloud(cloud_ref_filtered);
  icp.setInputTarget(cloud_target_filtered);
  icp.align(*cloud_reg);

  if (icp.hasConverged()) {
   save_matrix(icp, first, second);
  }
  *cloud_ref = *cloud_target;
 }
}
  \end{minted}
  \caption{ICP grupacija točaka - obrada oblaka}
  \label{coderef:voxel_grouping}
\end{listing}
 %ok

\chapter{Eksperimentalni rezultati}
\section{Generalizirani ICP algoritam}
Eksperimantalni rezultati za generalizirajući algoritam iteracije točaka su prikazani u sljedećim tablicama i grafovima. Grafovi i tablice se odnose na podskupove točaka primjera A i B. Zbog fizičke ograničenosti računala ICP algoritam je izvršen na manjem skupu podataka.

\subsubsection{tavblica ekspreimentalnih rezultata}
\begin{table}[H]
  \begin{tabular}{ |p{3cm}| |p{2cm}|p{2cm}|p{2cm}|p{2cm}| }
    \hline
    Rezultati& Primjer 1& Primjer 2&Primjer 3& Primjer 4\\
    \hline
    AMx [$m$]& 0.18427& 2.7368e-3& 1.41374e-3& 3.63932e-2\\
    AMy [$m$]&  16.3356& 1.38002e-3& 2.8428e-3& 1.32941e-2\\
    AMz [$m$]& 0.01657& 4.02492e-3& 5.811e-4& 8.78942e-4\\
    AMr [$rad$]& 1.53082e-3& 9.69565e-5& 4.95289e-5& 1.44264e-4\\
    AMp [$rad$]& 6.81777e-3& 8.68708e-5& 3.69849e-4& 2.43620e-4\\
    AMy [$rad$]& 5.63752e-2& 2.7926e-2& 1.07866e-2& 4.73278e-3\\
    \hline
    MSEx [$m^2$]& 3.9456e-2& 1.11415e-5& 3.99733e-6& 1.70555e-3\\
    MSEy [$m^2$]& 366.32160& 2.47215e-6& 1.61632e-5& 2.75735e-4\\
    MSEz [$m^2$]& 4.89075e-4& 2.15913e-5& 6.75381e-7& .98995e-7\\
    MSEr [$rad^2$]& 1.42492e-5& 1.17012e-8& 4.90623e-9& 5.0858e-8\\
    MSEp [$rad^2$]& 7.38958e-5& 1.38325e-8& 2.73577e-7& 8.0916e-8\\
    MSEy [$rad^2$]& 3.71623e-3& 1.05195e-3& 2.32704e-4& 2.94429e-5\\
    \hline
  \end{tabular}
  \caption{Usporedbe referentnih i estimiranih podataka}
  \label{res:ref_est_table}
\end{table}
\pagebreak
\subsubsection{Grafovi ekspreimentalnih rezultata}
\begin{figure}[H]
  \includegraphics[scale=0.4]{images/algo1/primjer3/usporedba_lokacija.png}
  \caption{Graf lokacije vozila}
  \label{eval:a1p3_lokacija}
\end{figure}
%ok
\chapter{Zaključak}
Prilikom lokalizacije vozila ili robota točnost podataka je vrlo važna. Koriste se podaci iz senzora, najčešće LIDAR-a da bi se odredio relativan položaj vozila. Ti podaci se prikazuju kao skupovi točaka. Ti podaci su vrlo jednostavni te iz je zato vrlo lako obrađivati u računalu. Ti podaci se obrađuju raznim algoritmima, ali naravno oni nisu u potpunosti točni. Točnost algoritama može biti promjenjena zbog okoline u kojoj se vozilo nalazi, šuma u očitanjima, broju reprezentativnih podataka u očitanju, samih grešaka u fizičkoj izvedbi snezora ili zbog načina na koji algoritam radi. Za evaluaciju točnosti algoritama za različite tipove skupova točaka se koristi simulator kao izvor referentnih podataka. U ovome radu su evaluirani generalizirajući ICP algoritam i ICP algoritam s prethodnim filtriranjem točaka. Rezultati su prikazani s grafovima te podacima o relativnoj i srednjoj pogrešci. Zbog same prirode simulatora, referentni podaci su garantirano točni te to omogućuje testiranje i drugih algoritama koji koriste senzore osim lasera. %ok

\bibliography{literatura}
\bibliographystyle{fer}

\begin{sazetak}
Postoje razni algoritmi za lokalizaciju vozila pomoću senzorskih očitanja. S obzirom da ne postoji savršen algoritam rezultati odstupaju od stvarnih vrijednosti. Koristeći simulator sa referentnim podacima možemo jednostavno usporediti stvarne podatke s rezultatima algoritama. Evaluiraju se generalizirajući algoritam iteracije najbliže točke te također taj isti algoritam ali s prijašnjim filtriranjem podataka. Rezultati su prikazani grafovima razlika koordinata i rotacija na poskupovima referentnih podataka. Također su izračunate srednje pogreške i prikazane tablično zbog jednostavnije usporedbe. Predstavljen je specifičan način evaluaciju algoritama korištenjem simulatora, ali čiji se koraci mogu općenito primjeniti na druge izvore podataka i druge algoritme.

\kljucnerijeci{simulacija; lokalizacija; iteracija najbližih točaka; Carla; evaluacija; PCL; oblak točaka}
\end{sazetak}
\pagebreak
% TODO: Navedite naslov na engleskom jeziku.
\engtitle{Autonomous vehicle localization in a simulated urban environment}
\begin{abstract}
There are various vehicle location algorithms using sensory readings. Since there is no perfect algorithm the results deviate from the actual values. Using a simulator with reference data, we can easily compare actual data with algorithm results. Evaluated algorithms are generalized iterative closest point and previous algorithm but with the previous data filtering. The results are shown using graphs of the coordinate differences and rotation differences between estimated values and referent values. Mean errors were also calculated and displayed in the table. A specific way of evaluating algorithms using a simulator was presented but whose steps generally be applied to other data sources and other algorithms.

\keywords{simulation; localization; iterative closest point; Carla; evaluation; PCL; point cloud}
\end{abstract}

\end{document}
